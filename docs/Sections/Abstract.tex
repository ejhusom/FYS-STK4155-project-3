\begin{abstract}

In this project we use tree-based and ensemble machine learning methods, specifically decision tree, bagging, random forest, AdaBoost, gradient boosting and XGBoost, on a human activity recognition (HAR) classification problem. The dataset consists of raw data from a triaxial, chest-mounted accelerometer, and was collected from 15 subjects performing 7 different activites. The goal is to use these methods to recognize what activity a subject performed based on the accelerometer data. When using a training set containing data from all 15 subjects (setting 1), all methods gave a test accuracy score above $97\%$, and the highest score, $99.5\%$, was obtained by both gradient boosting and XGBoost. When using separate subjects in the training and test set (setting 2), the highest score obtained was $19.5\%$, when using Gradient Boosting. The parameters of the models were tuned by using a cross-validation grid search, but overfitting of the training data proved to be a challenge. Improvement was made by reducing the depths of the trees, and XGBoost scored $60.5\%$ in setting 2 when using a depth of $1$. A simplified target set, with 4 activity categories, was also explored.
    
\end{abstract}