\section{Conclusion}\label{sec:Conclusion}

In this project we have studied the ability of tree-based and ensemble learning methods to recognize human activities based on accelerometer data. When using a training set containing data from all subjects (setting 1), both the single desicion tree, bagging, random forests and the three boosting methods gave a very high test accuracy score. The poorest performance was by the decision tree with $97.8\%$, and the best was by gradient boosting and XGBoost with $99.5\%$. Even though the methods can easily predict the target values when they are trained on all subjects, the overfitting problem becomes evident when testing the models on a set with separate subjects (setting 2). In this case the lowest score was $7.2\%$ by the decision tree, and the highest was $19.5\%$ by gradient boosting. These models were based on parameters tuned using a cross-validation grid search, which contributed greatly to the overfitting of the training data. Setting 2 is arguably the most realistic and interesting case, because it is much more useful to be able to use an activity recognition model on a subject without having to train the model specifically on data from said subject. The lack of generalization for models using setting 2 has also been experienced by other researchers\cite{Jordao2018} and seems to be a common challenge, because subjects perform the same activities in different ways. Our suggestion for future research is to experiment with using separate subjects for the training and validation set when performing cross-validation for parameter tuning, which might decrease the variance of the final result, and provide a more generalized model. We also explored some alternative ways of dealing with the overfitting problem, mainly by reducing the complexity of the models; this can be reviewed in the Appendix.