\section{Introduction}\label{sec:Introduction}

Human activity recognition (HAR) is technology that aims to recognize the activity that a person performs, based on observations of the person. Usually these observations are obtained by collecting data from various sensors placed on different parts of the body. This has many applications, for example in fields such as healthcare and sports science, and it is also widely used through the ubiquity of smartphones, smartwatches and other wearables that constantly track the movements of the user. One of the most common cases is recreational users who want to track their everyday activity, but this technology has also more critical applications, such as fall detection for sick and elderly\cite{Lustrek2009}. 

Activity recognition can be seen as a typical classification problem, with activity type as the target variable. While there are many possibilities with regards to what type of data to use as features of a model, accelerometer data are arguably one of the most attractive data types; accelerometers are easy to wear, cost-efficent, records data closely connected to the movements of a person, and are already present in most smartphones and wearables. In this project we deal with a dataset publicly available at the Machine Learning Repository of University of California, Irvine (UCI)\footnote{Link to the dataset at UCI: \href{https://archive.ics.uci.edu/ml/datasets/Activity+Recognition+from+Single+Chest-Mounted+Accelerometer}{https://archive.ics.uci.edu/ml/datasets/Activity+Recognition+from+Single+Chest-Mounted+Accelerometer}}, which provides data from a single chest-mounted accelerometer. Since activity sensors produce time series data, the predicted target for each observation will also be dependent on several of the preceeding data points. Many common machine learning methods, including the ones we explore in this project, will then require that we do a feature extraction on "windows" of the time series data, which has been done with success in previous research\cite{Casale2011}. Both the feature extraction and the training/test-split of the dataset have significant effect on the model performance, and this is one of the challenges we face in this project.

In this report we present the background theory and the methods used in the project. We have chosen to focus on tree-based methods, ranging from a single decision tree through bagging, random forests and several boosting algorithms, with trees as the base classifier. Our findings are presented in the results section, where we also include results from related research\cite{ravi} for the sake of comparison. This is followed by a discussion of the results, and lastly a conclusion. The appendix contains results from a deeper analysis of the dataset; these findings were obtained in the last stages of the project, where we explore additional methods to deal with overfitting of the training data.